\makeglossaries


\newglossaryentry{dbus}
{
	name=dbus,
	description={Desktop bus. Un sistema para habilitar la comunicación entre programas}
}

\newglossaryentry{Linux}
{
	name=Linux,
	description={Kernel creado por Linus Torvalds}
}
\newglossaryentry{GNU}
{
	name=Gnu is Not Unix,
	description={Proyecto con la finalidad de crear un sistema Unix-like completamente libre: GNU}
}
\newglossaryentry{GNU/Linux}
{
	name=GNU/Linux,
	description={Combinacón del kernel creado por Linus Torvalds y de los programas creados por el proyecto GNU}
}
\newglossaryentry{DFA}
{
	name=Doble factor de autenticación o DFA,
	description={Mayormente usado en cuentas online para verificar que realmente es la persona indicada quien accede a su cuenta. Consigue esto haciendo a la persona que se autentica en la cuenta pasar una prueba distinta a la contraseña, como verificación con SMS o por gódigo autogenerado}
}
\newglossaryentry{HAL}
{
	name=Hardware Abstraction Layer,
	description={Capa que se encarga de gestionar los nuevos dispositivos encontrados en el sistema para que el usuario no tenga que gestionarlo él mismo. Si no encuentra el driver, pregunta al usuario del sistema}
}
\newglossaryentry{udev}
{
	name=udev,
	description={Proporciona un directorio dinámico de dispositivos en el que elimina o añade nodos según los dispositivos que estén conectados. Generalmente se encuentran en /dev}
}
