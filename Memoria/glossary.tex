\makeglossaries


\newacronym{acr-usb}{USB}{Universal Serial Bus}
\newglossaryentry{USB}
{
	name=USB,
	description={Interfaz que permite la conexión de periféricos a diversos dispositivos}
}
\newglossaryentry{dbus}
{
	name=dbus,
	description={Un sistema para habilitar la comunicación entre programas}
}

\newglossaryentry{Linux}
{
	name=Linux,
	description={Kernel creado por Linus Torvalds}
}
\newacronym{acr-gnu}{GNU}{Gnu is Not Unix}
\newglossaryentry{GNU}
{
	name=GNU,
	description={Proyecto con la finalidad de crear un sistema Unix-like completamente libre: GNU}
}
\newglossaryentry{GNU/Linux}
{
	name=GNU/Linux,
	description={Combinacón del kernel creado por Linus Torvalds y de los programas creados por el proyecto GNU}
}
\newacronym{acr-hal}{HAL}{Hardware Abstraction Layer}
\newglossaryentry{HAL}
{
	name=HAL,
	description={Capa que se encarga de gestionar los nuevos dispositivos encontrados en el sistema para que el usuario no tenga que gestionarlo él mismo. Si no encuentra el driver, pregunta al usuario del sistema}
}
\newglossaryentry{udev}
{
	name=udev,
	description={Proporciona un directorio dinámico de dispositivos en el que elimina o añade nodos según los dispositivos que estén conectados. Generalmente se encuentran en /dev}
}
\newacronym{acr-pam}{PAM}{Pluggable Authentication Module}
\newglossaryentry{PAM}
{
	name=PAM,
	description={Módulo que funciona como adaptador entre los programas de verificación nuevos como por USB o llave maestra y los programas que gestionan la autenticación. Si no existiera, cada vez que se crea un nuevo esquema de autenticación, se debería de actualizar todos los programas que usan ese servicio}
}
\newglossaryentry{kernel}
{
	name=kernel,
	description={Intermediario entre el hardware y el software. Se encarga de gestionar la comunicación entre los programas y el hardware, de forma que si un programa quiere bajar el volumen de los altavoces, el programa se lo solicita al kernel y este al hardware mediante drivers}
}
\newacronym{acr-otp}{OTP}{One Time Password}
\newglossaryentry{OTP}
{
	name=OTP,
	description={Contraseña que solo es válida para un inicio de sesión o transacción. De esta forma aseguras que si alguien consigue entrar en tu cuenta, no pueda abusar de ella porque la contraseña cambia cada vez que inicias sesión}
}
\newglossaryentry{OATH}
{
	name=OATH,
	description={Estándar abierto que permite flujos simples de autorización de forma que un usuario puede permitir ver a una aplicación cierta información almacenada en ptra cuenta. Existen dos tipos de estándares, por tiempo y por evento}
}
\newglossaryentry{OpenPGP}
{
	name=Open PGP,
	description={Estándar ampliamente usado para firmar documentos de forma digital, cifrar y descifrar archivos}
}
\newglossaryentry{PAMUSB}
{
	name=PAMUSB,
	description={Módulo que implementa PAM escrito en C para añadir la opción de iniciar sesión mediante USB a sistemas/aplicaciones}
}
\newglossaryentry{udevil}
{
	name=udevil,
	description={Monta y desmonta USB y redes en el sistema sin necesidad de contrseña (set SUID). Incluye Devmon, un demonio para montar dispositivos al sistema automaticamente}
}
\newacronym{acr-acl}{ACL}{Access Control List}
\newglossaryentry{ACL}
{
	name=Access Control List,
	description={Modelo de permisos POSIX-compliant simple pero potente. Uno de las primeras formas de implementaciones de privilegios}
}
\newglossaryentry{SetUid}
{
	name=SetUid,
	description={Set user ID on execution: Si un archivo ejecutable contiene este bit, permite a usuarios ejecutar el archivo con los mismos privilegios que el usuario que posee el archivo}
}
\newglossaryentry{SetGid}
{
	name=SetGid,
	description={Set group ID on execution: Si un archivo ejecutable contiene este bit, permite a usuarios ejecutar el archivo con los mismos privilegios que el grupo que posee el archivo}
}
\newglossaryentry{StickyBit}
{
	name=StickyBit,
	description={Solo permite modificar el archivo/directorio por el usuario que lo ha posee}
}
\newacronym{acr-rbac}{RBAC}{Role Based Access Control}
\newglossaryentry{RBAC}
{
	name=Role Based Access Control,
	description={Sistema que trata de gestionar la seguridad de una forma basada en roles y no tan granular como ACL}
}
\newacronym{acr-ldap}{LDAP}{Lightweight Directory Access Protocol}
\newglossaryentry{LDAP}
{
	name=Lightweight Directory Access Protocol,
	description={Protocolo de acceso y búsqueda de datos a la base de datos distribuida X.500}
}
\newglossaryentry{X500}
{
	name=X.500,
	description={Es una base de datos distribuida que ofrece la posibilidad de buscar información por nombre (páginas blancas) y buscar información (Páginas amarillas)}
}
\newacronym{acr-mad}{MAD}{Microsoft Active Directory}
\newglossaryentry{MAD}
{
	name=Microsoft Active Directory,
	description={Protocolo de acceso y búsquedade datos en la base de datos distribuida X.500}
}
\newacronym{acr-sasl}{SASL}{Simple Authentication and Security Layer}
\newglossaryentry{SASL}
{
	name=Simple Authentication and Security Layer,
	description={forma de añadir autenticación y seguridad a protocolos basados en red}
}
\newglossaryentry{firewalls}
{
	name=Firewalls,
	description={Parte de un sistema o una red que está diseñada para bloquear el acceso no autorizado, permitiendo al mismo tiempo comunicaicones no autorizadas}
}
\newglossaryentry{kerberos}
{
	name=kerberos,
	description={Protocolo de autenticación de red}
}
\newacronym{acr-nis}{NIS}{Network Information Service}
\newglossaryentry{NIS/NIS+}
{
	name=Network Information Service,
	description={Base de datos distribuida en una red, que almacena las credenciales y permisos de los usuarios de la red}
}
\newacronym{acr-ssh}{SSH}{Secure SHell}
\newglossaryentry{SSH}
{
	name=SSH,
	description={Programa que permite conectarse a un sistema mediante una conexión segura}
}
\newacronym{acr-telnet}{Telnet}{Telecomunications Network}
\newglossaryentry{telnet}
{
	name=Telecomunication Network,
	description={Protocolo de red que nos permite aceder a otra máquina para manejarla remotamente como si estuviéramos sentados delante de ella. Tambien es el nombre de uno de los programas informáticos que implementan el protocolo}
}
\newacronym{acr-gui}{GUI}{Graphical User Interface}
\newglossaryentry{GUI}
{
	name=Graphical User Interface,
	description={Interfaz de usuario para comunicar el usuario con el ordenador mediante un conjunto de imágenes y objetos gráficos}
}
\newglossaryentry{bug}
{
	name=bug,
	description={Un bug, es un fallo en la línea de ejecución de un programa. ya sea lógico o sintáctico.}
}
\newacronym{acr-nist}{NIST}{National Institute of Standards and Technology}
\newglossaryentry{NIST}
{
	name=National Institute of Standards and Technology,
	description={Instituto que forma parte del departamento de comercio de Estados Unidos}
}
\newacronym{acr-dfa}{DFA}{Doble Factor de Autenticación}
\newglossaryentry{dfa}
{
	name=Doble Factor de Autenticación,
	description={Proporciona una capa más de seguridad. Es la combinación de dos de los siguientes factores: Conocimiento (Algo que sabe el usuario) Posesión (Algo que tiene el usuario) y herencia (algo que es el usuario)}
}
\newacronym{acr-bbp}{BBP}{Bug Bounty Program}
\newglossaryentry{bbp}
{
	name=Bug Bounty Program,
	description={Acuerdo al que llegan las empresas con los descubridores del error en la aplicación para recompensarle el hallazgo. Generalmente de forma económica}
}
\newglossaryentry{Ubuntu}
{
	name=Ubuntu,
	description={Distribución basada en GNU/Linux comercial gratuita}
}
\newglossaryentry{Manjaro}
{
	name=Manjaro,
	description={Distribución basada en Arch linux, que simplifica el proceso de instalación y ofrece varios programas instalados de forma predeterminada}
}
\newglossaryentry{Arch linux}
{
	name=Arch linux,
	description={Sistema basado en GNU/Linux que sigue una filosofía muy enfocada a la simplicidad}
}
\newglossaryentry{Git}
{
	name=Git,
	description={Software de control de versiones, creado para facilitar el desarrollo en el kernel de Linux. Ha sido adoptado por toda la comunidad de desarrolladores}
}
\newacronym{acr-bash}{BASH}{Bourne Again SHell}
\newglossaryentry{bash}
{
	name=bash,
	description={Es un programa informático diseñado para interpretar programas de consola. Es un proyecto creado por GNU}
}
\newglossaryentry{script}
{
	name=script,
	description={Programa informático que sigue una sintaxis específica creado para cumplir una funcion específica}
}
\newglossaryentry{GitHub}
{
	name=GitHub,
	description={Plataforma que implementa el cliente Git}
}
\newglossaryentry{C}
{
	name=C,
	description={C es un lenguaje de programación de bajo nivél con un paradigma imperativo}
}
\newglossaryentry{Python}
{
	name=Python,
	description={Python es un lenguaje de programación de alto nivel con paradigma funcional y orientado a objetos}
}
\newglossaryentry{Sun}
{
	name=SunSoft,
	description={Empresa tecnológica que propuso la implementación de PAM}
}
\newacronym{acr-api}{API}{Application Programming Interface}
\newglossaryentry{API}
{
	name=Application Programming Interface,
	description={Es un conjunto de métodos que facilitan la comunicación entre distintos tipos de programas. Facilitan la tarea al desarrollador porque este no se tiene que preocupar de implementar la lógica que devuelve el método implementado.}
}
\newglossaryentry{wifi}
{
	name=Wi-Fi,
	description={Red inalámbrica que permite la comunicación entre dispositivos, simulando una conexión por cable}
}
\newglossaryentry{fingerprint}
{
	name=fingerprint,
	description={En el campo de la seguridad informática, se le aplica este nombre al resultado del escaneo de las características, tanto hardware (número de serie) como software (versión del software, identificador) de un dispositivo o programa}
}
\newglossaryentry{bluetooth}
{
	name=Bluetooth,
	description={Estandar para la comunicación inalámbrica de corta distancia entre dispositivos}
}
\newglossaryentry{front-end}
{
	name=Front-end,
	description={Parte de un programa o aplicación que interactúa con el usuario}
}
\newglossaryentry{back-end}
{
	name=Back-end,
	description={Parte que implementa la lógica del programa. Generalmente, el usuario no interacciona con ella}
}
\newacronym{acr-spi}{SPI}{Service Provider Interface}
\newglossaryentry{SPI}
{
	name=Service Provider Interface,
	description={Conjunto de clases, interfaces, métodos, etc... que extiendes e implementas para conseguir una finalidad}
}
\newacronym{acr-tty}{TTY}{TeleTYpewriter}
\newglossaryentry{tty}
{
	name=tty,
	description={Los terminales antiguos, se conectaban al ordenador mediante teletipos. tty viene de la palabra TeleTYpewriter en ingés}
}
\newglossaryentry{root}
{
	name=root,
	description={La cuenta con mayor nivel de privilegios en GNU/Linux}
}
