\makeglossaries


\newglossaryentry{dbus}
{
	name=dbus,
	description={Desktop bus. Un sistema para habilitar la comunicación entre programas}
}

\newglossaryentry{Linux}
{
	name=Linux,
	description={Kernel creado por Linus Torvalds}
}
\newglossaryentry{GNU}
{
	name=Gnu is Not Unix,
	description={Proyecto con la finalidad de crear un sistema Unix-like completamente libre: GNU}
}
\newglossaryentry{GNU/Linux}
{
	name=GNU/Linux,
	description={Combinacón del kernel creado por Linus Torvalds y de los programas creados por el proyecto GNU}
}
\newglossaryentry{DFA}
{
	name=Doble factor de autenticación o DFA,
	description={Mayormente usado en cuentas online para verificar que realmente es la persona indicada quien accede a su cuenta. Consigue esto haciendo a la persona que se autentica en la cuenta pasar una prueba distinta a la contraseña, como verificación con SMS o por gódigo autogenerado}
}
\newglossaryentry{HAL}
{
	name=Hardware Abstraction Layer,
	description={Capa que se encarga de gestionar los nuevos dispositivos encontrados en el sistema para que el usuario no tenga que gestionarlo él mismo. Si no encuentra el driver, pregunta al usuario del sistema}
}
\newglossaryentry{udev}
{
	name=udev,
	description={Proporciona un directorio dinámico de dispositivos en el que elimina o añade nodos según los dispositivos que estén conectados. Generalmente se encuentran en /dev}
}
\newglossaryentry{PAM}
{
	name=Pluggable Authentication Module,
	description={Módulo que funciona como adaptador entre los programas de verificación nuevos como por USB o llave maestra y los programas que gestionan la autenticación. Si no existiera, cada vez que se crea un nuevo esquema de autenticación, se debería de actualizar todos los programas que usan ese servicio}
}
\newglossaryentry{kernel}
{
	name=kernel,
	description={Intermediario entre el hardware y el software. Se encarga de gestionar la comunicación entre los programas y el hardware, de forma que si un programa quiere bajar el volumen de los altavoces, el programa se lo solicita al kernel y este al hardware mediante drivers}
}
\newglossaryentry{OTP}
{
	name=One Time Password,
	description={Contraseña que solo es válida para un inicio de sesión o transacción. De esta forma aseguras que si alguien consigue entrar en tu cuenta, no pueda abusar de ella porque la contraseña cambia cada vez que inicias sesión}
}
\newglossaryentry{OATH}
{
	name=OATH,
	description={Estándar abierto que permite flujos simples de autorización de forma que un usuario puede permitir ver a una aplicación cierta información almacenada en ptra cuenta. Existen dos tipos de estándares, por tiempo y por evento}
}
\newglossaryentry{OpenPGP}
{
	name=Open PGP,
	description={Estándar ampliamente usado para firmar documentos de forma digital, cifrar y descifrar archivos}
}
\newglossaryentry{PAMUSB}
{
	name=PAMUSB,
	description={Módulo que implementa PAM escrito en C para añadir la opción de iniciar sesión mediante USB a sistemas/aplicaciones}
}
