\documentclass[titlepage]{article}
\usepackage[utf8]{inputenc}
\usepackage{hyperref}
\usepackage{csquotes}
\usepackage{listings}
\usepackage{graphicx}
\usepackage{comment}
\usepackage[
backend=biber,
style=alphabetic,
sorting=ynt]{biblatex}
 
\usepackage[nonumberlist,acronymlists={gloss}]{glossaries}
\makeglossaries

\newglossaryentry{latex}
{
	name=latex,
	description={A markup language}
}


\addbibresource{bibliography.bib}
\title{Doble factor de autenticación independiente de SO}
\author{Roselló Morell, Sergio\\
\texttt{sergio.rosello@live.u-tad.com}}

\begin{document}

\maketitle
\tableofcontents

\begin{abstract}
El siguiente documento contiene el desarrollo de mi Trabajo de fin de grado el cual se centra en proporcionar una forma de bloquear o desbloquear un ordenador con arquitectura basada en Linux sin necesidad de contraseña. Para conseguir esto, voy a usar un demonio que reside entre el sistema operativo y el kernel. Al ser de tan bajo nivel, este programa, en principio debería ser independiente de sistema operativo. El único requisito es que el ordenador utilice \Gls{dbus} \cite{dbus}. 
\end{abstract}
\section{Investigación}
%\cite{codecoffee}
Para \Gls{latex} poder afrontar este desafío, antes necesito averiguar exactamente como funciona Linux a bajo nivel. Una de las formas de averiguar eso es prueba de que compila...
\clearpage
\printglossaries
\clearpage
\printbibliography[heading=bibintoc,title={Bibliografía}]
\end{document}



















\begin{comment}

DBUS (Desktop bus)
------------------------------------------------------------
- Inter-process communication mechanism
- Fast and lightweight (Unified middleware UNDERNEATH free OS)
- Non-transactional, statefull, connection-based
- Supports both one-to-one and publish/subscribe communication (similar to TCP and UDP)
- Two major components
	- point-to-point communication dbus library
	- dbus daemon
		- A kind of street that messages are transported over.
		- Any number of processes may be connected at a given time.
- PROCESSES CONNECT TO THE DAEMON USING THE LIBRARY (only use for the library)
- Normally two buses are running in the system
	- System (Miscellaneous system-wide communication (eg: new piece of hardware hooked up))
	- Session bus (carries traffic (Normally) under a singlel user identity)
- Addresses
	- Every bus has an address describing how to connect to it
- Connections
	- Every connection can be addressed to that bus under one or more names. (Connection bus names)
	- Bus names consist of a series of identifiers separated by dots. (com.acme.foo)
- Objects
	- An object is created by a client process and exists within the context of that client's connection to the bus
	- Object is a way for the client process to offer it's services on the bus but one client may create any number of objects
	- Bus imposes an object-centric view of communications (3 types)
		- Requests sent to objects by client processes
		- Replies to requests (Going from an object back to a requesting process)
		- One way messages emanating from objects, broadcast to any connected clients that have registered an interest in them. (Two forms of communication: "1:1 request-reply" going to an object and "1:n publish:subscribe" coming from an object)

\end{comment}

